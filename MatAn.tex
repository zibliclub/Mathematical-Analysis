\documentclass{article}
\usepackage[utf8]{inputenc}
\usepackage[russian]{babel}
\usepackage{setspace,amsmath}
\usepackage{amsthm}
\usepackage{amsfonts}

\theoremstyle{definition}
\newtheorem{definition}{Определение}[section]
\newtheorem{example}{Пример}
\newtheorem{statement}{Утверждение}[section]
\newtheorem*{remark}{Замечание}

\begin{document}

\section{Первообразная и неопределенный интеграл}

\begin{definition}[Первообразная функция]
  Пусть \(X\) - промежуток, \(f:X\rightarrow \mathbb{R}\). Функция \(F(x)\) называется
  \textbf{первообразной} \(f(x)\), если производная \(F'(x) = f(x)\), при этом \(F(x)\)
  дифференцируема и непрерывна.
\end{definition}

\begin{example}
  \(f(x) = 2x \implies F(x) = x^{2}\). В самом деле, \(F'(x) = f(x)\).
\end{example}

\begin{statement}
  (О первообразной)
  \begin{enumerate}
    \item Если \(F(x)\) - первообразная функции \(f(x)\) на промежутке \(X\), и
    \(\varPhi(x) = F(x) + C, \ c \in \mathbb{R}\), то \(\varPhi(x)\) - тоже первообразная.
    \item Если \(F(x)\) и \(\varPhi(x)\) - две первообразные для \(f(x)\) на промежутке \(X\),
    то \(\exists C = const, \ c \in \mathbb{R}\) такая, что \(\varPhi(x) = F(x) + C\). 
  \end{enumerate}
\end{statement}

\begin{proof}
  (Утверждения о первообразной)
  \begin{enumerate}
    \item \(\Phi'(x) = (F(x) + C)' = F'(x) = f(x) \implies \Phi(x)\)
      - первообразная для \(f(x)\).
    \item Так как \(F(x)\) и \(\Phi(x)\) - первообразные для \(f(x)\), то \(F'(x) = f(x),
    \Phi'(x) = f(x)\). Рассмотрим функцию \(\phi = \Phi(x) - F(x), \ \forall x \in X\):
    \(\phi'(x) = \Phi'(x) - F'(x) = f(x) - f(x) = 0\). Рассмотрим
    \(\forall x_{1}, x_{2} \in X\), по теореме Лагранжа, \(\exists \xi \in (x_{1}, x_{2}):
    \ \phi(x_{1}) - \phi(x_{2}) = \phi'(\xi)(x_{1} - x_{2}) = 0 \implies \phi(x_{1}) =
    \phi(x_{2}) \implies \phi(x) = const\) для \(\forall x \in X\).
  \end{enumerate}
\end{proof}

\begin{definition}[Неопределенный интеграл]
  Совокупность всех первообразных для функции \(f(x)\) на промежутке \(X\) называется
  \textbf{неопределенным интегралом} и обозначается:
  \begin{equation*}
    \int f(x) dx
  \end{equation*}
  Таким образом, \(\int f(x) dx = \{F(x) + C\), где \(F'(x) = f(x), \ C \in \mathbb{R}\}\),
  или:
  \begin{equation*}
    \int f(x) dx = F(x) + C
  \end{equation*}
\end{definition}

\begin{remark}
  (Для неопределенного интеграла)
  \begin{itemize}
    \item \((\int f(x) dx)_{x}' = (F(x) + C)_{x}' = F'(x) = f(x)\);
    \item \(d(\int f(x) dx) = d(F(x) + C) = (F(x) + C)' dx = F'(x) dx = f(x) dx\);
    \item \(\int d(F(x)) = \int F'(x) dx = \int f(x) dx = F(x) + C, \ C \in \mathbb{R}\). 
  \end{itemize}
\end{remark}

\begin{definition}[Интегрирование]
  Операция нахождения первообразной функции \(f(x)\) называетсвя ее \textbf{интегрированием}.
\end{definition}

\begin{statement}
  (Основные методы интегрирования) \\
  Пусть \(f: X \rightarrow \mathbb{R}, \ g:X \rightarrow \mathbb{R}, \ X\) - промежуток:
  \begin{enumerate}
    \item Пусть \(\alpha, \beta \in \mathbb{R} = const\), тогда:\\
      \(\int (\alpha f(x) + \beta g(x)) dx = \alpha \int f(x) dx + \beta \int g(x) dx\).
    \item Формула интегрирования по частям: \\
      \(udv = uv - \int udv, \ u = u(x), v = v(x)\).
    \item Интегрирование подстановкой: \\
      Пусть \(T\) - промежуток, \(X = X(t)\) - дифференцируема на \(T\). \\
      Тогда \(\int f(X(t)) * X'(t) dt = F(X(t)) + C = \int f(x) dx + C\).
  \end{enumerate}
\end{statement}

\begin{proof}
  (Утверждения об основных методах интегрирования)
  \begin{enumerate}
    \item Возьмем производную по \(x\) от обеих частей равенства: \(\int(\alpha f(x) +
      \beta g(x))_{x}' = \alpha f(x) + \beta g(x) = \alpha F'(x) + \beta G'(x) =
      \alpha(\int f(x) dx)_{x}' + \beta(\int g(x) dx)_{x}'\) - является производной для
      \(\alpha \int f(x) dx + \beta \int g(x) dx\).
    \item Рассмотрим \(d(uv) = v du + u dv: \ \int d(uv) = \int v du + \int u dv\).
      Так как \(d(uv) = uv\), то из того, что \(\int d(uv) = \int v du + \int u dv
      \implies \int u dv = uv - \int v du\).
    \item \(f(X(t)) * X'(t) dt = \int f(X(t)) dx(t) = \int f(x) dx = F(x) + C =
      F(X(t)) + C; \ (F(X(t)) + C)_{t}' = F_{t}' * X_{t}' = f(x) * X'(t) =
      (\int f(X(t)) * X'(t) dt)_{t}'\).
  \end{enumerate}
\end{proof}

\begin{example}
  (Интегрирование функций)
  \begin{enumerate}
    \item \(\int x^{3} dx = \frac{x^{4}}{4} + C\)
    \item \(\int \ln x dx = 
      \left(
        \begin{array}{c}
          u = \ln x, \ dv = dx, \ du = d(\ln x) = \frac{dx}{x} \implies \\
          \implies \int dv = \int dx \implies v = x
        \end{array}
      \right) = x \ln x - \int x \frac{dx}{x} = x \ln x - x + C\)
    \item \(\int \sqrt{1 - x^{2}} dx =
    \left(
      \begin{array}{c}
        x = \sin t, \ dx = \\
        = d(\sin t) = \cos t dt
      \end{array}
    \right) = \int \cos^{2} t dt = \int \frac{1}{2}(1 + \cos 2t) dt = \frac{1}{2} \int dt + \frac{1}{2} \int \cos 2t dt =
    \frac{t}{2} + \frac{1}{4} \int \cos 2t d(2t) = \frac{t}{2} + \frac{1}{4} \sin 2t + C = \frac{\arcsin x}{2} +
    \frac{1}{2} x \sqrt{1 - x^{2}} + C\)
  \end{enumerate}
\end{example}

\begin{example}
  (Неинтегрируемые функции)
  \begin{equation*}
    \int \frac{x}{\ln x} dx; \quad \int \frac{e^{x}}{x} dt; \quad \int e^{x^{2}} dx
  \end{equation*}
\end{example}

\clearpage

\subsection{Интегрирование рациональных дробей}

\begin{definition}[Рациональная дробь]
  Функция вида \(\frac{P(x)}{Q(x)}\), где \(P(x), \ Q(x)\) - многочлены, называется \textbf{рациональной дробью},
  или рациональной функцией. \\
  Если \(\deg P(x) < \deg Q(x)\), то дробь называется \textbf{правильной}, иначе - \textbf{неправильной}. \\
  Если дробь \(\frac{P(x)}{Q(x)}\) - неправильная, то ее можно представить в виде \(\frac{P(x)}{Q(x)} = M(x) +
  \frac{P_{1}(x)}{Q_{1}(x)}\), где \(\frac{P_{1}(x)}{Q_{1}(x)}\) - правильная дробь. Поэтому достаточно уметь
  интегрировать правильную дробь.
\end{definition}

\begin{definition}[Простые дроби]
  \textbf{Простыми дробями} будем называть дроби следующих четырех видов:
  \begin{enumerate}
    \item \(\frac{A}{x - a}, \quad A,a \in \mathbb{R}\)
    \item \(\frac{A}{(x - a)^{k}}, \quad A,a \in \mathbb{R}, \ k > 1\)
    \item \(\frac{Ax + B}{x^{2} + px + q}, \quad A,B,p,q \in \mathbb{R}, \ p^{2} - 4q < 0\)
    \item \(\frac{Ax + B}{(x^{2} + px + q)^{k}}, \quad A,B,p,q \in \mathbb{R}, \ k > 1, \ p^{2} - 4q < 0\)
  \end{enumerate}
\end{definition}


\end{document}