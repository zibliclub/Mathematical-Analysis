\documentclass{report}
\usepackage[utf8]{inputenc}
\usepackage[russian]{babel}
\usepackage{setspace,amsmath}
\usepackage{amssymb}
\usepackage{amsthm}
\usepackage{amsfonts}
\usepackage{scalerel}
\usepackage{graphicx}
\usepackage{float}
\usepackage{wrapfig}
\usepackage[unicode, pdftex]{hyperref}

\def\stretchint#1{\vcenter{\hbox{\stretchto[440]{\displaystyle\int}{#1}}}}
\def\scaleint#1{\vcenter{\hbox{\scaleto[3ex]{\displaystyle\int}{#1}}}}

\theoremstyle{definition}
\newtheorem{definition}{Определение}[section]
\newtheorem{example}{Пример}
\newtheorem*{effect}{Следствие}
\newtheorem{statement}{Утверждение}[section]
\newtheorem*{remark}{Замечание}
\newtheorem{lemma}{Лемма}[section]
\newtheorem{theorem}{Теорема}[section]

\title{Математический Анализ \\ 2 семестр}
\author{Данил Заблоцкий}
\date{\today}

\begin{document}

\maketitle
\tableofcontents
\chapter{Дифференциальное исчисление}

\section{Первообразная и неопределенный интеграл}

\begin{definition}[Первообразная функция]
  Пусть \(X\) - промежуток, \(f:X\rightarrow \mathbb{R}\). Функция \(F(x)\) называется
  \textbf{первообразной} \(f(x)\), если производная \(F'(x) = f(x)\), при этом \(F(x)\)
  дифференцируема и непрерывна.
\end{definition}

\begin{example}
  \(f(x) = 2x \implies F(x) = x^{2}\). В самом деле, \(F'(x) = f(x)\).
\end{example}

\begin{statement}
  (О первообразной)
  \begin{enumerate}
    \item Если \(F(x)\) - первообразная функции \(f(x)\) на промежутке \(X\), и
          \(\varPhi(x) = F(x) + C, \ c \in \mathbb{R}\), то \(\varPhi(x)\) - тоже первообразная.
    \item Если \(F(x)\) и \(\varPhi(x)\) - две первообразные для \(f(x)\) на промежутке \(X\),
          то \(\exists C = const, \ c \in \mathbb{R}\) такая, что \(\varPhi(x) = F(x) + C\).
  \end{enumerate}
\end{statement}

\begin{proof}
  (Утверждения о первообразной)
  \begin{enumerate}
    \item \(\Phi'(x) = (F(x) + C)' = F'(x) = f(x) \implies \Phi(x)\)
          - первообразная для \(f(x)\).
    \item Так как \(F(x)\) и \(\Phi(x)\) - первообразные для \(f(x)\), то \(F'(x) = f(x),
          \Phi'(x) = f(x)\). Рассмотрим функцию \(\phi = \Phi(x) - F(x), \ \forall x \in X\):
          \(\phi'(x) = \Phi'(x) - F'(x) = f(x) - f(x) = 0\). Рассмотрим
          \(\forall x_{1}, x_{2} \in X\), по теореме Лагранжа, \(\exists \xi \in (x_{1}, x_{2}):
          \ \phi(x_{1}) - \phi(x_{2}) = \phi'(\xi)(x_{1} - x_{2}) = 0 \implies \phi(x_{1}) =
          \phi(x_{2}) \implies \phi(x) = const\) для \(\forall x \in X\).
  \end{enumerate}
\end{proof}

\begin{definition}[Неопределенный интеграл]
  Совокупность всех первообразных для функции \(f(x)\) на промежутке \(X\) называется
  \textbf{неопределенным интегралом} и обозначается:
  \begin{equation*}
    \int f(x) dx
  \end{equation*}
  Таким образом, \(\int f(x) dx = \{F(x) + C\), где \(F'(x) = f(x), \ C \in \mathbb{R}\}\),
  или:
  \begin{equation*}
    \int f(x) dx = F(x) + C
  \end{equation*}
\end{definition}

\begin{remark}
  (Для неопределенного интеграла)
  \begin{itemize}
    \item \((\int f(x) dx)_{x}' = (F(x) + C)_{x}' = F'(x) = f(x)\);
    \item \(d(\int f(x) dx) = d(F(x) + C) = (F(x) + C)' dx = F'(x) dx = f(x) dx\);
    \item \(\int d(F(x)) = \int F'(x) dx = \int f(x) dx = F(x) + C, \ C \in \mathbb{R}\).
  \end{itemize}
\end{remark}

\begin{definition}[интегрирование]
  Операция нахождения первообразной функции \(f(x)\) называется ее \textbf{интегрированием}.
\end{definition}

\begin{statement}
  (Основные методы интегрирования) \\
  Пусть \(f: X \rightarrow \mathbb{R}, \ g:X \rightarrow \mathbb{R}, \ X\) - промежуток:
  \begin{enumerate}
    \item Пусть \(\alpha, \beta \in \mathbb{R} = const\), тогда:\\
          \(\int (\alpha f(x) + \beta g(x)) dx = \alpha \int f(x) dx + \beta \int g(x) dx\).
    \item Формула интегрирования по частям: \\
          \(udv = uv - \int udv, \ u = u(x), v = v(x)\).
    \item Интегрирование подстановкой: \\
          Пусть \(T\) - промежуток, \(X = X(t)\) - дифференцируема на \(T\). \\
          Тогда \(\int f(X(t)) * X'(t) dt = F(X(t)) + C = \int f(x) dx + C\).
  \end{enumerate}
\end{statement}

\begin{proof}
  (Утверждения об основных методах интегрирования)
  \begin{enumerate}
    \item Возьмем производную по \(x\) от обеих частей равенства: \(\int(\alpha f(x) +
          \beta g(x))_{x}' = \alpha f(x) + \beta g(x) = \alpha F'(x) + \beta G'(x) =
          \alpha(\int f(x) dx)_{x}' + \beta(\int g(x) dx)_{x}'\) - является производной для
          \(\alpha \int f(x) dx + \beta \int g(x) dx\).
    \item Рассмотрим \(d(uv) = v du + u dv: \ \int d(uv) = \int v du + \int u dv\).
          Так как \(d(uv) = uv\), то из того, что \(\int d(uv) = \int v du + \int u dv
          \implies \int u dv = uv - \int v du\).
    \item \(f(X(t)) * X'(t) dt = \int f(X(t)) dx(t) = \int f(x) dx = F(x) + C =
          F(X(t)) + C; \ (F(X(t)) + C)_{t}' = F_{t}' * X_{t}' = f(x) * X'(t) =
          (\int f(X(t)) * X'(t) dt)_{t}'\).
  \end{enumerate}
\end{proof}

\begin{example}
  (Интегрирование функций)
  \begin{enumerate}
    \item \(\int x^{3} dx = \frac{x^{4}}{4} + C\)
    \item \(\int \ln x dx =
          \left|
          \begin{array}{c}
            u = \ln x, \ dv = dx, \ du = d(\ln x) = \frac{dx}{x} \implies \\
            \implies \int dv = \int dx \implies v = x
          \end{array}
          \right| = x \ln x - \int x \frac{dx}{x} = x \ln x - x + C\)
    \item \(\int \sqrt{1 - x^{2}} dx =
          \left|
          \begin{array}{c}
            x = \sin t, \ dx = \\
            = d(\sin t) = \cos t dt
          \end{array}
          \right| = \int \cos^{2} t dt = \int \frac{1}{2}(1 + \cos 2t) dt = \frac{1}{2} \int dt + \frac{1}{2} \int \cos 2t dt =
          \frac{t}{2} + \frac{1}{4} \int \cos 2t d(2t) = \frac{t}{2} + \frac{1}{4} \sin 2t + C = \frac{\arcsin x}{2} +
          \frac{1}{2} x \sqrt{1 - x^{2}} + C\)
  \end{enumerate}
\end{example}

\begin{example}
  (Неинтегрируемые функции)
  \begin{equation*}
    \int \frac{x}{\ln x} dx; \quad \int \frac{e^{x}}{x} dt; \quad \int e^{x^{2}} dx
  \end{equation*}
\end{example}

\subsection{Интегрирование рациональных дробей}

\begin{definition}[Рациональная дробь]
  Функция вида \(\frac{P(x)}{Q(x)}\), где \(P(x), \ Q(x)\) - многочлены, называется \textbf{рациональной дробью},
  или рациональной функцией. \\
  Если \(\deg P(x) < \deg Q(x)\), то дробь называется \textbf{правильной}, иначе - \textbf{неправильной}. \\
  Если дробь \(\frac{P(x)}{Q(x)}\) - неправильная, то ее можно представить в виде \(\frac{P(x)}{Q(x)} = M(x) +
  \frac{P_{1}(x)}{Q_{1}(x)}\), где \(\frac{P_{1}(x)}{Q_{1}(x)}\) - правильная дробь. Поэтому достаточно уметь
  интегрировать правильную дробь.
\end{definition}

\begin{definition}[Простые дроби]
  \textbf{Простыми дробями} будем называть дроби следующих четырех видов:
  \begin{enumerate}
    \item \(\frac{A}{x - a}, \quad A,a \in \mathbb{R}\)
    \item \(\frac{A}{(x - a)^{k}}, \quad A,a \in \mathbb{R}, \ k > 1\)
    \item \(\frac{Ax + B}{x^{2} + px + q}, \quad A,B,p,q \in \mathbb{R}, \ p^{2} - 4q < 0\)
    \item \(\frac{Ax + B}{(x^{2} + px + q)^{k}}, \quad A,B,p,q \in \mathbb{R}, \ k > 1, \ p^{2} - 4q < 0\)
  \end{enumerate}
\end{definition}

\subsection{Интегрирование рациональных дробей}

\begin{enumerate}
  \item \(\int \frac{A}{x - a} dx = A \int \frac{d(x-a)}{x-a} =
        \left|
        \begin{array}{c}
          \int \frac{dt}{t} \\
          d(x-a) = dx
        \end{array}
        \right| = A \ln |x - a| + C\)
  \item \(\int \frac{A}{(x-a)^{k}} dx = A \int (x - a)^{-k} dx = A \int (x-a)^{-k} d(x-a) =
        \left|
        \begin{array}{c}
          \int t^{n}dt = \\
          = \frac{t^{n+1}}{n+1}
        \end{array}
        \right| = A \frac{(x-a)^{-k + 1}}{-k + 1} + C = \frac{A}{(x-a)^{k-1}(1-k)} + C\)

        \clearpage
  \item \(\int \frac{Ax + B}{x^{2} + px + q} dx =
        \left|
        \begin{array}{c}
          x^{2} + px + q = (x^{2} + 2\frac{p}{2}x + \frac{p^{2}}{4}) - \frac{p^{2}}{4} + q = \\
          = (x + \frac{p}{2})^{2} - \frac{p^{2} - 4q}{4}, \ (- \frac{p^{2} - 4q}{4} = C > 0)
        \end{array}
        \right| = \int \frac{Ax + B}{(x + \frac{p}{2})^{2} + C} dx = A \int \frac{x dx}{(x + \frac{p}{2})^{2}
          + C} + B \int \frac{dx}{(x + \frac{p}{2})^{2} + C} =
        \left|
        \begin{array}{c}
          d((x + \frac{p}{2})^{2} + C) = \\
          = 2(x + \frac{p}{2} dx)
        \end{array}
        \right| = \ldots\) \\

        \begin{center}
          \begin{minipage}{8cm}
            \setlength{\parindent}{0em}
            \(A \int \frac{x dx}{(x + \frac{p}{2})^{2} + C} = \frac{A}{2} \int \frac{(2(x + \frac{p}{2})-p)dx}
            {(x + \frac{p}{2})^{2} + C} = \frac{A}{2} \int \frac{2(x + \frac{p}{2})dx}{(x + \frac{p}{2})^{2} + C} -
            \frac{Ap}{2} \int \frac{dx}{(x + \frac{p}{2})^{2} + C} =
            \left|
            \begin{array}{c}
              \int \frac{dx}{(x + \frac{p}{2})^{2} + C} = I
            \end{array}
            \right| = \frac{A}{2} \int \frac{d((x + \frac{p}{2})^{2} + C)}{(x + \frac{p}{2})^{2} + C} -
            \frac{Ap}{2} I = \frac{A}{2} \ln |(x + \frac{p}{2})^{2} + C| - \frac{Ap}{2}I;\) \\

            \(I = \frac{dx}{(x + \frac{p}{2})^{2} + C} = \frac{1}{C} \int \frac{\sqrt{C}d(\frac{x}{\sqrt{C}} +
              \frac{p}{2 \sqrt{C}})}{(\frac{x}{\sqrt{C}} + \frac{p}{2\sqrt{C}})^{2} + 1} =
            \left|
            \begin{array}{c}
              \int \frac{dt}{t^{2} + 1} = \arctan t + C
            \end{array}
            \right| = \frac{1}{\sqrt{C}} \arctan(\frac{x + 2p}{2\sqrt{C}}) + C_{1};\) \\

            \(\frac{1}{C} (x + \frac{p}{2})^{2} = (\frac{1}{\sqrt{C}})^{2} (x + \frac{p}{2})^{2} =
            (\frac{1}{\sqrt{C}} (x + \frac{p}{2}))^{2} = (\frac{x}{\sqrt{C} + \frac{p}{2\sqrt{C}}})^{2} \); \\
          \end{minipage}
        \end{center}

        \(\ldots = \frac{A}{2} \ln |(x + \frac{p}{2})^{2} - \frac{p^{2} - 4q}{4}| + (B - \frac{Ap}{2})
        \frac{1}{\sqrt{C}} \arctan (\frac{x + 2p}{2 \sqrt{C}}) + C_{1}\)

  \item \(\int \frac{Ax + B}{(x^{2} + px + q)^{k}}dx =
        \left|
        \begin{array}{c}
          d(x^{2} + px + q) = \\
          = 2x + p
        \end{array}
        \right| = \int \frac{\frac{A}{2}(2x + p) + B - \frac{Ap}{2}}{(x^{2} + px + q)^{k}}dx =
        \frac{A}{2} \int \frac{d(x^{2} + px + q)}{(x^{2} + px + q)^{k}} + (B - \frac{Ap}{2}) \int
        \frac{dx}{((x + \frac{p}{2})^{2} + (\frac{-p^{2} + 4a}{4}))^{k}} = \frac{A}{2(1-k)}
        \frac{1}{(x^{2} + px + q)^{k - 1}} + \frac{(B - \frac{Ap}{2})}{(- \frac{p^{2} + 4q}{a})^{k}}
        \scaleint{8ex} \frac{dx}{((\frac{x + \frac{p}{2}}{\sqrt{\frac{-p^{2} + 4q}{4}}})^{2} + 1)^{k}} =
        \frac{A}{2(1 - k)} \frac{1}{(x^{2} + px + q)^{k-1}} + \frac{(B-\frac{Ap}{2})
        \frac{\sqrt{-p^{2} + 4q}}{2}}{(\frac{-p^{2} + 4q}{4})^{k}} \scaleint{10ex} \frac{d\Big(
          \frac{x + \frac{p}{2}}{\frac{\sqrt{-p^{2} + 4q}}{2}}\Big)}{\Bigg(\Big(\frac{x + \frac{p}{2}}
          {\frac{\sqrt{-p^{2} + 4q}}{2}}\Big)^{2} + 1\Bigg)^{k}}\)\\

        Таким образом, чтобы вычислить интеграл 4., нужно вычислить интеграл \(\int \frac{dt}{(t^{2} + 1)^{k}} =
        \left|
        \begin{array}{c}
          u = \frac{1}{(t^{2} + 1)^{k}}; \ du = d((t^{2} + 1)^{k}) = -k (t^{2} + 1)^{-k-1}2t dt \\
          dv = dt \implies v = t
        \end{array}
        \right| = \frac{t}{(t^{2} + 1)^{k}} - \int \frac{-2kt^{2}}{(t^{2} + 1)^{k+1}}dt = \frac{t}{(t^{2} + 1)^{k}}
        + 2k(\int \frac{t^{2} + 1}{(t^{2} + 1)^{k+1}}dt - \int \frac{dt}{(t^{2} + 1)^{k + 1}}); \\
        \int \frac{dt}{(t^{2} + 1)^{k}} = \frac{t}{(t^{2} + 1)^{k}} + 2k \int \frac{dt}{(t^{2} + 1)^{k}}
        - 2k \int \frac{dt}{(t^{2} + 1)^{k+1}}
        \left|
        \begin{array}{c}
          \frac{dt}{(t^{2} + 1)^{k}} = I_{k} \\
          \frac{dt}{(t^{2} + 1)^{k+1}} = I_{k+1}
        \end{array}
        \right|\); \\
        \(2k I_{k+1} = \frac{t}{(t^{2} + 1)^{k}} + (2k - 1)I_{k}; \quad I_{k+1} = \frac{t}{2k(t^{2} + 1)^{k}}
        + \frac{2k - 1}{2k}I_{k}, \ k = 1, \ldots\)
\end{enumerate}

\clearpage

\subsection{Разложение рациональной дроби на простые}

\begin{lemma}
  Пусть \(\frac{P(x)}{Q(x)}\) - правильная рациональная дробь (несократимая). Причем \(Q(x) = (x - a)^{k}
  Q_{1}(x)\), где \(Q_{1}(x)\) не делится на \((x - a)\). Тогда \(\exists\) многочлен \(P_{1}(x)\) из
  \(\exists A \in \mathbb{R} : \ \frac{P(x)}{Q(x)} = \frac{A}{(x - a)^{k}} + \frac{P_{1}(x)}
  {(x-a)^{k-1} Q_{1}(x)}\). При этом дробь (рациональная) \(\frac{P_{1}(x)}{(x - a)^{k-1}Q_{1}(x)}\)
  - правильная.
\end{lemma}

\begin{proof}
  Рассмотрим $\frac{A}{(x - a)^{k}} + \frac{P_{1}(x)}{(x - a)^{k-1}Q_{1}(x)} = \frac{Q_{1}(x)A +
      (x-a)P_{1}(x)}{Q(x)}$. Нужно доказать, что $\frac{Q_{1}(x)A + (x-a)P_{1}(x)}{Q(x)} = \frac{P(x)}{Q(x)}$.\\

  Отсюда следует, что для выполнения леммы, многочлен $P(x)$ должен расскладываться: $P(x) = Q_{1}(x) +
    (x-a)P_{1}(x) \implies P_{1}(x) = \frac{P(x) - AQ_{1}(x)}{x - a}$. Чтобы существовал многочлен $P_{1}(x)$,
  нужно, чтобы $P(x) - AQ_{1}(x)$ делилась на $x - a$. Для этого точка $a$ должна быть корнем $P(x) - AQ_{1}(x)$,
  то есть чтобы $P(a) - AQ_{1}(a) = 0 \implies A = \frac{P(a)}{Q(a)}; \quad Q_{1}(a) \ne 0$ по условию.
  Таким образом, при $A = \frac{P(a)}{Q_{1}(a)}$, функция $P_{1}(x)$ будет являться многочленом
  $P_{1}(x) = \frac{P(x) - \frac{P(a)}{Q_{1}(a)}Q_{1}(x)}{x-a}$. \\

  Покажем, что дробь $\frac{P_{1}(x)}{(x-a)^{k-1}Q_{1}(x)}$ - правильная, то есть $\deg P_{1}(x) <
    \deg[(x-a)^{k-1}Q_{1}(x)]$. Имеем, $P_{1}(x) = \frac{P(x) - AQ_{1}(x)}{x-a}; \quad \deg P_{1}(x)
    \leqslant \max(\deg P(x), \\ \deg Q_{1}(x)) - 1$. Тогда $\deg P_{1}(x) \leqslant \deg P(x) - 1 <
    \deg Q(x) - 1 = \deg[(x-a)^{k-1}Q_{1}(x)]$. \\

  Если $\deg Q_{1}(x) \geqslant \deg P(x) \implies \deg P_{1}(x) \leqslant \deg Q_{1}(x) - 1 < \deg Q(x) - 1 =
    \deg [(x-a)^{k-1}Q_{1}(x)]$. Дробь $\frac{P_{1}(x)}{(x - a)^{k-1}Q_{1}(x)}$ - правильная.
\end{proof}

\begin{lemma}
  Пусть $\frac{P(x)}{Q(x)}$ - правильная дробь. При этом $Q(x) = (x^{2} + px + q)^{k} Q_{1}(x)$, здесь
  $p^{2} - 4q < 0$. Тогда $\exists M, N \in \mathbb{R}$ и $\exists$ многочлен $P_{1}(x): \\
    \frac{P(x)}{Q(x)} = \frac{Mx + N}{(x^{2} + px + q)^{k}} + \frac{P_{1}(x)}{(x^{2} + px + q)^{k-1}Q_{1}(x)}$.
  При этом $Q_{1}(x)$ не делится на $x^{2} + px + q$. Дробь $\frac{P_{1}(x)}{(x^{2} + px + q)^{k-1}Q_{1}(x)}$
  - правильная.
\end{lemma}

\begin{proof}
  Если разложение $\frac{P(x)}{Q(x)} = \frac{Mx + N}{(x^{2} + px + q)^{k}} + \frac{P_{1}(x)}
    {(x^{2} + px + q)^{k-1}Q_{1}(x)}$ верно, то: $\frac{P(x)}{Q(x)} = \frac{(Mx + N)Q_{1}(x) + P_{1}(x)
      (x^{2} + px + q)}{Q(x)}$, следовательно $P(x)$ должен выражаться как: $P(x) = (Mx + N)Q_{1}(x) +
    P_{1}(x)(x^{2} + px + q) \implies P_{1}(x) = \frac{P(x) - (Mx + N)Q_{1}(x)}{x^{2} + px + q}$. \\

  Так как нужно, чтобы $P_{1}(x)$ был многочленом, то $P(x) - (Mx + N)Q_{1}(x)$ должно делиться на
  $x^{2} + px + q$. \\

  Рассмотрим остаток от деления $P(x)$ на $x^{2} + px + q$ в форме $\alpha x + \beta$ и остаток от деления
  $Q_{1}(x)$ на $x^{2} + px + q$ в форме $\gamma x + \delta$. \\

  Таким образом, $P(x) = (x^{2} + px + q) P_{2}(x) + (\alpha x + \beta); \quad Q_{1}(x) = (x^{2} + px + q)
    Q_{2}(x) + (\gamma x + \delta)$. \\

  Отсюда достаточно показать, что на $x^{2} + px + q$ делится многочлен $\alpha x + \beta - (Mx + N)
    (\gamma x + \delta) = -M\gamma x^{2} + x(-N\gamma - M\delta + \alpha) + (\beta - N\delta)$. \\

  Поделим полученный выше многочлен на $x^{2} + px + q: \\ \frac{-M\gamma x^{2} + x(-N\gamma - M\delta + \alpha)
      + (\beta - N\delta)}{x^{2} + px + q} = -M\gamma + (\alpha - N\gamma - M\delta + M\gamma p)x + \\
    (\beta - N\delta + M\gamma q)$. \\

  Для целого деления необходимо, чтобы: \\
  \begin{equation*}
    \left\{
    \begin{array}{ll}
      \alpha - N\gamma - M\delta + M\gamma p = 0 \\
      \beta - N\delta + M\gamma q = 0
    \end{array}
    \right. \implies
    \left\{
    \begin{array}{ll}
      -(\delta - \gamma p)M - \gamma N = - \alpha \\
      \gamma qM - \delta N = -\beta
    \end{array}
    \right.
  \end{equation*}
  где $M, N$ - неизвестные; \\

  $\left\{
    \begin{array}{ll}
      \alpha - N\gamma - M\delta + M\gamma p = 0 \\
      \beta - N\delta + M\gamma q = 0
    \end{array}
    \right. ; \
    \left|
    \begin{array}{cc}
      \delta - \gamma p & \gamma  \\
      \gamma q          & -\delta \\
    \end{array}
    \right| = -\delta^{2} + \gamma p\delta - \gamma^{2}q$. \\

  Заметим, что $\alpha$ и $\beta$, а так же $\gamma$ и $\delta$ одновременно в 0 не обращаются.\\
  $p^{2} - 4q < 0 \implies q \ne 0, \quad -(\delta^{2} + \gamma^{2}q) + \gamma p \delta$:
  \begin{enumerate}
    \item $\gamma = 0,\ \delta = 0$ -- невозможно;
    \item $\gamma = 0,\ \delta \ne 0 \implies -\delta^{2} \ne 0$;
    \item $\gamma \ne 0,\ \delta = 0 \implies -\gamma^{2}q \ne 0$;
    \item $\gamma \ne 0,\ \delta \ne 0$.
  \end{enumerate}

  Тогда, если $-(\delta^{2} + \gamma^{2}q) + \gamma p \delta = 0 \implies \gamma p \delta = \delta^{2}
    + \gamma^{2}q; \\ p^{2} - 4q < 0, \ p^{2} < 4q \implies 0 \leqslant \frac{p^{2}}{4} < q$ \\

  $\gamma \ne 0$: если $(\frac{\delta}{\gamma})^{2} + (-\frac{\delta}{\gamma})p + q = 0$, то
  $x = \frac{\delta}{\gamma}$ - корень многочлена $x^{2} + px + q \implies$ противоречие с тем,
  что $x^{2} + px + q$ не имеет вещественных корней $\implies \Delta \ne 0 \implies \exists M,N$
  и $\exists$ многочлен $P_{1}(x)$.
\end{proof}

\subsection{Метод неопределенных коэф-ов (следствия лемм)}

Если $\frac{P(x)}{Q(x)}$ - правильная дробь и $Q(x) = (x-a_{1})^{k_{1}} * \ldots * (x - a_{s})^{k_{s}} * (x^{2} + p_{1}x +
  q_{1})^{m_{1}} * \ldots * (x^{2} + p_{r}x + q_{r})^{m_{r}}$, то верно следующее разложение:

\begin{center}
  {\Large $\frac{P(x)}{Q(x)} = \sum_{i = 0}^{k_{1}-1} \frac{A_{i}}{(x - a_{i})^{k_{1} - i}} + \ldots + \sum_{i = 0}^{k_{s} - 1}
      \frac{A^{s}_{i}}{(x - a_{s})^{k_{s} - i}} + \sum_{i = 0}^{m_{1} - 1} \frac{M_{i}x + N_{i}}{(x^{2} + p_{1}x + q_{1})
        ^{m_{1} - i}} + \ldots + \sum_{i = 0}^{m_{r} - 1} \frac{M_{i}^{r}x + N_{i}^{r}}{(x^{2} + p_{r}x + q_{r})^{m_{r} - i}}$, \\}
\end{center}

где $A_{i}, \ldots, A_{i}^{s}, \quad M_{i}, N_{i}, \ldots, M_{i}^{r}, N_{i}^{r} \in \mathbb{R}$.

\begin{example}
  $Q(x) = (x-1)^{3}(x+2)^{2}(x^{2} + x + 1)^{3}$
  \begin{center}
    {\Large $\frac{x^{5} - x^{3} + 1}{Q(x)} = \frac{A^{1}_{0}}{(x - 3)^{3}} + \frac{A^{1}_{1}}{(x - 3)^{2}} +
        \frac{A^{1}_{2}}{(x - 3)} + \frac{A^{2}_{0}}{(x + 2)^{2}} + \frac{A^{2}_{1}}{(x + 2)} + \frac{M_{0}x + N_{0}}
        {(x^{2} + x + 1)^{3}} + \frac{M_{1}x + N_{1}}{(x^{2} + x + 1)^{2}} + \frac{M_{2}x + N_{2}}{(x^{2} + x + 1)}$}
  \end{center}
\end{example}

Приведем в $\frac{P(x)}{Q(x)} = \sum_{i = 0}^{k_{1}-1} \frac{A_{i}}{(x - a_{i})^{k_{1} - i}} + \ldots$
правую часть к общему знаменателю и получим: $\frac{P(x)}{Q(x)} \equiv \frac{R(x)}{Q(x)}; \quad \deg Q(x) =
  k_{1} + \ldots + k_{s} + 2m_{1} + \ldots + 2m_{r} = n$; \\

$l = \deg R(x) = \deg P(x) \leqslant \deg Q(x) - 1$. \\

Количество неизвестных коэф. у множества $R(x)$ равно $n$ штук, приравнивая коэф. при соответствующих степенях
$x$ (в том числе при $x^{0}$) получим $n$ уравнений с $n$ неизвестными (старшая степень $x$ множества $R(x)$ равна $n-1$).

\subsection{Метод Остроградского}

\begin{theorem}
  Пусть $\frac{P(x)}{Q(x)}$ - правильная несократимая дробь. \\ Тогда $\int \frac{P(x)}{Q(x)}dx = \frac{P_{1}(x)}{Q_{1}(x)}
    + \int \frac{P_{2}(x)}{Q_{2}(x)}dx$. Дроби $\frac{P_{1}(x)}{Q_{1}(x)}$ и $\frac{P_{2}(x)}{Q_{2}(x)}$ - правильные.
  $Q(x) = Q_{1}(x) Q_{2}(x)$ и многочлен $Q_{2}(x)$ представляет собой произведение всех линейных и квадратичных множителей
  многочлена $Q(x)$, взятых в первой степени.
\end{theorem}

\begin{example}
  $\int \frac{x^{2} + 2x + 5}{(x-2)(x^{2} + 1)^{2}}dx = \frac{P_{1}(x)}{x^{2} + 1} + \int \frac{P_{2}(x)}{(x-2)(x^{2}+1)}dx
    = \frac{Ax + B}{x^{2} + 1} + \int \frac{Cx^{2} + Dx + E}{(x-2)(x^{2}+1)}dx$
\end{example}

\begin{proof}
  Рассмотрим $\int \frac{A}{(x-a)^{k}} dx = \frac{A}{1 - k} \frac{1}{(x-a)^{k-1}}$; \\

  $\int \frac{Mx + N}{(x^{2} + px + q)^{k}} dx = \frac{A}{(x^{2} + px + q)^{k-1}} + B\int \frac{dx}{(x^{2}+px+q)^{k}} =
    \frac{A}{(x^{2} + px + q)^{k-1}} + \frac{C}{(x^{2} + px + q)^{k-1}} + D\int \frac{dx}{(x^{2} + px + q)^{k-1}} =
    \frac{A}{(x^{2} + px + q)^{k-1}} + \ldots + \frac{V}{(x^{2} + px + q)^{2}} + W\int \frac{dx}{x^{2} + px + q}$.\\

  Представим $Q(x)$ в виде $Q(x) = (x-a_{1})^{k_{1}} * \ldots * (x-a_{s})^{k_{s}} * (x^{2} + p_{1}x + q_{1})^{m_{1}} *
    \ldots * (x^{2} + p_{r}x + q_{r})^{m_{r}}$, тогда: \\

  $Q_{2}(x) = (x-a_{1}) * \ldots * (x-a_{s}) * (x^{2} + p_{1}x + q_{1}) * \ldots * (x^{2} + p_{r}x + q_{r})$;\\

  $Q_{1}(x) = (x-a_{1})^{k_{1}-1} * \ldots * (x-a_{s})^{k_{s}-1} * (x^{2} + p_{1}x + q_{1})^{m_{1}-1} * \ldots *
    (x^{2} + p_{r}x + q_{r})^{m_{r}-1}$;\\

  Из метода неопределенных коэффициентов и того, что $\int \frac{Mx + N}{(x^{2} + px + q)^{k}} dx = \frac{A}
    {(x^{2} + px + q)^{k-1}} + \ldots + \frac{V}{(x^{2} + px + q)^{2}} + W\int \frac{dx}{x^{2} + px + q} \implies
    \int \frac{P(x)}{Q(x)} dx = \frac{P_{1}(x)}{Q_{1}(x)} + \int \frac{P_{2}(x)}{Q_{2}(x)} dx$. \\

  Как найти $P_{1}(x)$ и $Q_{1}(x)$?\\

  Продифференцируем $\int \frac{P(x)}{Q(x)}dx = \frac{P_{1}(x)}{Q_{1}(x)} + \int \frac{P_{2}(x)}{Q_{2}(x)}dx$: \\

  $\frac{P(x)}{Q(x)} = \frac{P_{1}'(x) Q_{1}(x) - P_{1}(x)Q_{1}'(x)}{Q_{1}^{2}} + \frac{P_{2}(x)}{Q_{2}(x)}$.
  Рассмотрим: $\frac{P_{1}'(x) Q_{1}(x) - P_{1}(x)Q_{1}'(x)}{Q_{1}^{2}} = \frac{P_{1}'(x) - P_{1}(x)
      \frac{Q_{1}'(x)}{Q_{1}(x)}}{Q_{1}(x)} = \frac{P_{1}'(x)Q(x) - P_{1}(x)\frac{Q_{1}'(x)Q_{2}(x)}{Q_{1}(x)}}
    {Q_{1}(x)Q_{2}(x)}$.\\

  Пусть $H(x) = \frac{Q_{1}'(x)Q_{2}(x)}{Q_{1}(x)}$ - многочлен (нужно показать).\\

  Пусть $Q_{1}(x)$ имеет среди своих множителей многочлен вида $(x-a)^{n}$, тогда $Q_{1}'(x)$ будет иметь в своем
  составе $(x-a)^{n-1}$, а $Q_{2}(x)$ только содержит в себе выражение $(x-a) \implies H(x)$ - многочлен.\\

  Коэффициенты многочленов $P_{1}(x)$ и $P_{2}(x)$ можно найти с помощью метода неопределенных коэффициентов
  из выражения $\frac{P(x)}{Q(x)} = \frac{Mx + N}{(x^{2} + px + q)^{k}} + \frac{P_{1}(x)}{(x^{2} + px + q)^{k-1}Q_{1}(x)}$.
\end{proof}

\chapter{Интегральное исчисление}

\section{Интеграл Римана}

\begin{definition}[интеграл Римана]
  Пусть $f:[a;b]\rightarrow\mathbb{R}$. Разобьем отрезок $[a;b]$ на $n$ частей точками $a = x_{0} < x_{1} < \ldots <
    x_{n-1} < x_{n} = b$. В каждом таком кусочке выберем точку $\xi_{i} \in [x_{i-1};x_{i}], \ i = 1,\ldots ,n$.\\

  \begin{figure}[H]
    \begin{center}
      \includegraphics[scale=0.7]{graph1.png}\label{figure1}
    \end{center}
  \end{figure}

  $\Delta i = [x_{i-1}, x_{i}], \quad \Delta x = x_{i} - x_{i-1}$ - длина отрезка $\Delta i$.\\

  Составим сумму $S_{n} = \sum_{i=1}^{n} f(\xi_{i})\Delta x_{i}$, где $f(\xi_{i})$ - высота $i$-го прямоугольника и
  $\Delta x_{i}$ - ширина $i$-го прямоугольника.\\

  $S_{n}$ - площадь ступенчатой фигуры, составленной из прямоугольников под графиком функции $f(x)$.\\

  Говорят, что функция $f$ интегрируема на $[a;b]$, если существует предел интегральных сумм $S_{n}$, то есть
  $\exists \underset{max\Delta x_{i}\rightarrow0}{\lim}S_{n}$, причем этот предел не зависит ни от способа разбиения
  отрезка $[a;b]$, ни от способа выбора точек $\xi_{i}$.\\

  Этот предел называется \textbf{интегралом Римана} функции $f$ на $[a;b]$. Класс интегрируемых функций на отрезке
  $[a;b]$ будем обозначать $R([a;b])$.
\end{definition}

\section{Базы. Предел функции по базе}

\begin{definition}[база множества]
  Пусть $X$ - произвольное множество.\\

  Система $\beta$ подмножеств множества $X$ называется \textbf{базой} на $X$, если:
  \begin{enumerate}
    \item $\forall \beta \in \beta \quad \beta \ne \o$
    \item $\forall \beta_{1}, \beta_{2} \in \beta \ \exists \beta_{3} \in \beta: \quad \beta_{3} \subset \beta_{1}
            \cap \beta_{2}$
  \end{enumerate}
\end{definition}

\begin{example}[баз множества]
  \begin{enumerate}
    \item $\beta = \{X\}$ - база
    \item $X = \mathbb{R}, \quad \beta = \{\beta_{n} = (-\frac{1}{n};\frac{1}{n}), \ n \in \mathbb{N}\}$
    \item $X = \mathbb{R}, \quad \beta = \{\beta_{\epsilon} = \{x: \  0 < |x| < \epsilon\}, \epsilon > 0\}$
          (выколотые окрестности нуля)
  \end{enumerate}
\end{example}

\begin{definition}[предел по базе]
  Пусть $f:X\rightarrow\mathbb{R}, \ \beta$ - база на $X$\\

  Число $A\in\mathbb{R}$ называется \textbf{пределом} функции $f$ \textbf{по базе $\beta$}, если
  $\forall \epsilon > 0 \ \exists$ элемент базы $\beta \in \beta: \quad |f(x) - A| < \epsilon$.
  \begin{equation*}
    \underset{\beta}{\lim} f(x)
  \end{equation*}
\end{definition}

\begin{definition}[предел по базе (МП)]
  Пусть $(Y, d)$ - МП, $f:X\rightarrow Y, \ \beta$ - база на $X$.\\

  $y\in Y$ называется \textbf{пределом} функции $f(x)$ \textbf{по базе $\beta$}, если $\forall \epsilon > 0 \
    \exists \beta \in \beta \ \forall x \in \beta: \quad d(f(x), y) < \epsilon$, или, что то же самое,
  $\forall V_{Y}(y) \ \exists \beta \in \beta \quad f(\beta) \subset V_{Y}(y)$, где $V_{Y}$ - окрестность
  метрического пространства $Y$.
\end{definition}

\begin{theorem}[основные свойства предела по базе]
  Пусть $f:X\rightarrow\mathbb{R}, \ \beta$ - база на $X$:
  \begin{enumerate}
    \item Если $\exists \underset{\beta}{\lim}f(x)$, то $\exists \beta \in \beta: \ f$ ограничена на $\beta$
    \item Если $\underset{\beta}{\lim}f(x) = A$ и $\underset{\beta}{\lim}f(x) = B$, то $A = B$
  \end{enumerate}
\end{theorem}

\clearpage

\begin{theorem}[связь предела по базе с арифметическими операциями]
  Пусть $f:X\rightarrow\mathbb{R}, \ g:X\rightarrow\mathbb{R}, \ \beta$ - база на $X, \quad \underset{\beta}
    {\lim}f(x) = A, \quad \underset{\beta}{\lim}g(x) = B$:
  \begin{enumerate}
    \item $\exists \underset{\beta}{\lim}(f(x)\pm g(x)) = A\pm B$
    \item $\exists \underset{\beta}{\lim}(f(x)g(x)) = AB$
    \item $\exists \underset{\beta}{\lim}(\frac{f(x)}{g(x)}) = \frac{A}{B}$, если $g(x)\ne 0, \ \beta \ne 0$
  \end{enumerate}
\end{theorem}

\begin{theorem}[связь предела функции по базе с неравенствами]
  Пусть $f:X\rightarrow\mathbb{R}, \ g:X\rightarrow\mathbb{R}, \ \beta$ - база на $X$:
  \begin{enumerate}
    \item Если $\exists\beta\in\beta: \quad \forall x \in \beta \ f(x) \leqslant g(x)$, то $\underset{\beta}
            {\lim}f(x) \leqslant \underset{\beta}{\lim}g(x)$
    \item Если $\underset{\beta}{\lim}f(x) < \underset{\beta}{\lim}g(x)$, то $\exists\beta\in\beta \ \forall
            x \in \beta \quad f(x) < g(x)$\\

          Если $\underset{\beta}{\lim}f(x) \geqslant \underset{\beta}{\lim}g(x)$, то $\exists\beta\in\beta \ \forall
            x \in \beta \quad f(x) \geqslant g(x)$
    \item Если $h:X\rightarrow \mathbb{R}$ и $\exists\beta\in\beta: \ \forall x \in \beta \ f(x) \leqslant h(x)
            \leqslant g(x)$ \textbf{И} $A = \underset{\beta}{\lim}f(x) = \underset{\beta}{\lim}g(x)$, то $\underset{\beta}
            {\lim}h(x) = A$
  \end{enumerate}
\end{theorem}

\begin{theorem}[критерий Коши существования предела по базе]
  Существуют две формулировки:
  \begin{enumerate}
    \item Пусть $f:X\rightarrow\mathbb{R}, \ \beta$ - база на $X$.

          Функция $f(x)$ имеет предел по базе $\beta \iff \forall \epsilon > 0 \ \exists\beta\in\beta: \quad \forall
            x_{1},x_{2} \in \beta \ |f(x_{1}) - f(x_{2})| < \epsilon$

    \item Пусть $(Y,d)$ - МП (\textbf{полное}), $f:X\rightarrow Y, \ \beta$ - база на $Y$.

          Функция $f(x)$ имеет предел по базе $\beta \iff \forall \epsilon>0\exists\beta\in\beta: \quad \forall
            x_{1},x_{2} \in \beta \ d(f(x_{1}), f(x_{2})) < \epsilon$
  \end{enumerate}
\end{theorem}

\begin{proof}
  (критерия Коши $\exists$ предела по базе)\\

  $"\rightarrow"$ Пусть $\exists \underset{\beta}{\lim}f(x) = A$. Покажем, что $\forall \epsilon > 0 \exists \beta \in
    \beta: \quad \forall x_{1},x_{2} \in \beta \ |f(x_{1}) - f(x_{2})| < \epsilon$. Рассмотрим $| f(x_{1}) - f(x_{2}) |
    = | f(x_{1} - A) + (A - f(x_{2})) | \leqslant | f(x_{1}) - A | + | f(x_{2}) - A | < \frac{\epsilon}{2} +
    \frac{\epsilon}{2} = \epsilon$.\\

  $"\leftarrow"$ Пусть $\forall \epsilon > 0 \ \exists \beta\in\beta: \quad \forall x_{1},x_{2}\in\beta \ | f(x_{1}) -
    f(x_{2}) |<\epsilon$. Покажем, что $\exists \underset{\beta}{\lim}f(x)$. Возьмем $\beta_{1} \in \beta: \quad
    \forall x_{1},x_{2}\in\beta_{1} \ | f(x_{1}) - f(x_{2}) | < 1$. Возьмем $\beta_{1}'\in\beta: \quad \forall
    x_{1},x_{2}\in\beta_{1}' \ | f(x_{1}) - f(x_{2}) | < \frac{1}{2}$. Пусть $\beta_{2} \subset \beta_{1} \cap \beta_{1}'$
  и так далее.\\

  Таким образом построим систему вложенных множеств: $\beta_{1} \supset \beta_{2} \supset \ldots \supset \beta_{n}
    \supset \ldots$, при этом $\forall x_{1},x_{2} \in \beta_{n} \ | f(x_{1}) - f(x_{2}) | < \frac{1}{2^{n-1}}$.
  Воспользуемся полнотой пространства, то есть в нем $\exists \lim f(x)$, если $f(x)$ - фундаментальная.\\

  $\forall n \in \mathbb{N}$ рассмотрим $x_{n} \in \beta_{n}$. Тогда, если $n < m \ (m \in \mathbb{N})$, то
  для $x_{n} \in \beta_{n}$ и $x_{m} \in \beta_{n} \ | f(x_{n}) - f(x_{m}) | < \frac{1}{2^{n-1}}$.\\

  Таким образом последовательность ${f(x_{n})}$ - фундаментальная $\implies \\ \exists \underset{n\rightarrow\infty}{\lim}
    f(x_{n}) = A$. Покажем, что $A = \underset{\beta}{\lim}f(x)$. Пусть $\epsilon > 0$ задано. Выберем $n\in\mathbb{N}:\quad
    \frac{1}{2^{n-1}} < \frac{\epsilon}{2}$. Возьмем $m>n:\quad | f(x_{m}) - A | < \frac{\epsilon}{2}$. Возьмем $\beta =
    \beta_{n}$. Тогда $\forall x \in \beta \ | f(x) - A | = | f(x) - f(m) + f(x_{m}) - A | \leqslant | f(x) - f(x_{m}) | +
    | f(x_{m}) - A | < \frac{\epsilon}{2} + \frac{\epsilon}{2} = \epsilon$.\\

  Следовательно, $\exists\underset{\beta}{\lim}f(x) = A$.
\end{proof}

\section{Разбиение. Интеграл Римана (v.2)}

\begin{definition}[разбиение]
  Пусть дан отрезок $[a;b]$. \textbf{Разбиением} $P$ отрезка $[a;b]$ называется набор точек $a=x_{0}<x_{1}<\ldots
    <x_{n-1}<x_{n}=b$. То есть $P = \{x_{0},\ldots,x_{n}\}$. Отрезки $[x_{i-1};x_{i}] = \Delta_{i}$.
  $x_{i}-x_{i-1} = \Delta x_{i}$ - длина $i$-го отрезка разбиения $\lambda(P) = \underset{i=\overline{0,n}}
    {\max}\{\Delta x_{i}\}$. Величины $\Delta_{i}, \Delta x_{i}, \lambda(P)$ - параметры ограничения.
\end{definition}

\begin{definition}[разбиение с отмеченными точками]
  \textbf{Разбиением с отмеченными точками} называется пара наборов
  \begin{equation*}
    P(\xi) = \{x_{0},\ldots,x_{n}\},\{\xi_{0},\ldots,\xi_{n}\},
  \end{equation*}
  \center{где $a=x_{0} < \ldots < x_{n}=b, \ \xi_{i}\in [x_{i-1};x_{i}]$.}
  \begin{figure}[h]
    \begin{minipage}[h]{0.49\linewidth}
      \center{\includegraphics[scale=0.6]{graph2.png}}
    \end{minipage}
    \hfill
    \begin{minipage}[h]{0.49\linewidth}
      Пусть $\Re_{\xi} = \{(P,\xi)\}$ - семейство всевозможных разбиений с отмеченными точками отрезка $[a,b]$.
    \end{minipage}
  \end{figure}
\end{definition}

Рассмотрим $\beta_{\delta} = \{(P,\xi):\quad \lambda(P) < \delta\}, \ \beta_{\delta}\subset P_{\xi}$:

\begin{statement}
  Множество $\beta = \{\beta_{\delta}:\quad \delta > 0\}$ является базой на $\Re_{\xi}$.
\end{statement}

\begin{proof}
  (утверждения 2.3.1.).
  \begin{enumerate}
    \item $\forall\delta>0 \ \beta_{\delta}$ - непусто.\\

          В самом деле, пусть отрезок $[a;b]$ поделен на $n$ равных частей, причем $n$ выбирается из соображений,
          чтобы $\Delta x_{i} = \Delta x \quad \forall i = \overline{1,n} \ (1,\ldots,n), \ \Delta x < \delta$.\\

          Пусть $\xi_{i} \in [x_{i-1};x_{i}]$ - середины отрезков $[x_{i-1};x_{i}]$.

    \item Покажем, что $\forall \beta_{\delta_{1}},\beta_{\delta_{2}} \in \beta \ \exists \beta_{\delta_{3}}
            \subset \beta_{\delta_{1}} \cap \beta_{\delta_{2}}$.\\

          Пусть заданы $\delta_{1}>0, \delta_{2}>0$. Покажем, что $\exists\beta_{3}>0$:\\
          $\beta_{\delta_{3}} \subset \beta_{\delta_{1}} \cap \beta_{\delta_{2}}$. Если $\delta_{1} < \delta_{2}$,
          то $\delta_{3} = \delta_{1}$ или $\delta_{3} = \frac{\delta_{1}}{2}$.
  \end{enumerate}
\end{proof}

\begin{definition}[!]
  Пусть $f:[a;b]\rightarrow\mathbb{R}, \ (P,\xi)$ - разбиение отрезка $[a;b]$ с отмеченными точками. Составим сумму:
  \begin{equation*}
    \sigma(f,(P,\xi)) = \sum_{k=1}^{n}f(\xi_{k})\Delta x_{k}
  \end{equation*}
  Можно смотреть на $\sigma$ для фиксированной функции $f(x)$ как на функцию, сопоставляющую разбиение $(P,\xi)\in
    \Re_{\xi}$ сумме $\sum_{k=1}^{n}f(\xi_{k})\Delta x_{k}$, то есть $\sigma_{f}:\Re_{\xi}\rightarrow\mathbb{R}$
  (то есть $(P,\xi)$ - аргумент функции $\sigma$).\\

  Говорят, что функция $f:[a;b]\rightarrow\mathbb{R}$ интегрируема по Риману на $[a;b]$, если:
  \begin{equation*}
    \exists\underset{\lambda(P)\rightarrow0}{\lim}\sigma_{f}((P,\xi)) = \underset{\lambda(P)\rightarrow0}{\lim}
    \sum_{k=1}^{n}f(\xi_{k})\Delta x_{k}
  \end{equation*}
  Или, что то же самое, если $\forall\epsilon>0 \ \exists\delta>0$ и соответствующий элемент $\beta_{\delta} \in
    \beta:\quad \forall$ разбиения $(P,\xi): \ \lambda(P) < \delta$ выполняется неравенство \\ $| \sigma_{f}
    ((P,\xi)) - I | < 0$:
  \begin{equation*}
    I = \underset{\lambda(P)\rightarrow0}{\lim}\sigma_{f}((P,\xi)) = \int_{a}^{b}f(x)dx
  \end{equation*}
\end{definition}

Обозначим базу $\beta$ из утверждения 2.3.1. как $\lambda(P)\rightarrow0$.

\begin{theorem}[необходимое условие интегрируемости функции]
  $*\_*$

  Если $f:[a;b]\rightarrow\mathbb{R}$ интегрируема на $[a;b]$ (то есть $f\in\mathbb{R}[a;b]$), то \\ $f$
  ограничена на $[a;b]$.
\end{theorem}

\begin{proof}
  От противного:

  Допустим, что $f$ интегрируема на $[a;b]$, но неограничена, то есть: $\forall M>0 \ \exists x \in
    [a;b]: \quad | f(x) | > M$. Покажем, что функция $\sigma ((P,\xi))$ не имеет предела по базе на $[a;b]$.

  То есть $\exists \epsilon > 0: \ \forall \delta > 0 \ \exists (P',\xi')$ и $(P'',\xi''): \quad
    \lambda(P') < \delta, \ \lambda(P'')<\delta \ (\lambda(P'') = \max\Delta x_{i})$, но
  $(\sigma(P'',\xi'') - \sigma(P'',\xi'')) \geqslant \epsilon$.

  Положим, $\epsilon = 1$. Пусть $\delta > 0$ задана. Выберем разбиение с отмеченными точками $(P',\xi')$
  такое, что $\lambda(P')<\delta, \ P' = \{a = x_{0},x_{1},\ldots,x_{n}=b\}, \ \epsilon_{i} \in [x_{1},
    \ldots,x_{n}]$. Поскольку функция $f$ неограничена на $[a;b]$, то существует хотя бы один элемент
  разбиения $[x_{i-1},x_{i}] = \Delta i:$ функция $f$ неограничена на (? Спасибо Максим). В качестве
  $P^{n}$ возьмем $P', \ \xi'' = \{\xi_{1}',\xi_{2}',\ldots,\xi_{i}'',\ldots,\xi_{n}'\}, \
    \lambda(P'')<\delta$ и $|f(\xi_{i}'') - f(\xi_{i}')| > \frac{1}{\Delta x_{i}}$. Разбиения $P'$ и $P''$
  совпадают, точки разбиения так же совпадают, кроме $\xi_{i}''$.

  Рассмотрим $| \sigma((P'',\xi'')) - \sigma((P',\xi')) | = | \sum_{k=1}^{n}\Delta x_{k}f(\xi_{k}') -
    \sum_{k=1}^{n}\Delta x_{k}f(\xi_{k}'') | = | \Delta x_{i}(f(\xi_{i}'') - f(\xi_{i}')) | >
    \frac{\Delta x_{i}}{\Delta x_{i}} = 1 = \epsilon$.
\end{proof}

\section{Критерий интегрируемости}

\subsection{Суммы Дарбу}

\begin{definition}[нижняя/верхняя суммы Дарбу]
  Пусть $f[a;b] \rightarrow \mathbb{R}, \ P$ - произвольное разбиение отрезка $[a;b]$. Числа
  $\underline{S}(P) = \sum_{k=1}^{n}m_{k}\Delta x_{k}$ и $\overline{S}(P) = \sum_{k=1}^{n}M_{k}
    \Delta x_{k}$, где $m_{k} = \underset{\xi \in \Delta k}{\inf}f(\xi), \ M_{k} = \underset{\xi\in\Delta k}
    {\sup}f(\xi)$, называются \textbf{нижней} и \textbf{верхней суммами Дарбу}, отвечающими разбиению $P$.

  \begin{figure}[h]
    \begin{minipage}[h]{0.49\linewidth}
      \center{\includegraphics[scale=0.6]{graph4.png}}
    \end{minipage}
    \hfill
    \begin{minipage}[h]{0.49\linewidth}
      \center{\includegraphics[scale=0.6]{graph3.png}}
    \end{minipage}
  \end{figure}
\end{definition}

\begin{theorem}[свойства сумм Дарбу]
  Свойства:
  \begin{enumerate}
    \item $\forall(P,\xi) \ \underline{S}(P)\leqslant\sigma_{f}((P,\xi))\leqslant\overline{S}(P)$
    \item Если разбиение $P'$ получено из разбиения $P$ добавлением новых точек, то $\underline{S}
            (P') \geqslant \underline{S}(P)$ и $\overline{S}(P')\leqslant\overline{S}(P)$
    \item $\forall P_{1},P_{2} \quad \underline{S}(P_{1})\leqslant\overline{S}(P_{2})$
  \end{enumerate}
\end{theorem}

\begin{proof}
  (теоремы 2.4.1)

  \begin{enumerate}
    \item $\underline{S}(P) = \sum_{k=1}^{n}m_{k}\Delta x_{k} \leqslant \sum_{k=1}^{n}f(\xi_{k})
            \Delta x_{k} \leqslant \sum_{k=1}^{n}M_{k}\Delta x_{k} = \overline{S}(P)$, где $f(\xi_{k}) =
            \sigma((P,\xi))$, вроде
    \item Пусть $P$ - произвольное разбиение отрезка $[a;b]$. Построим $P'$. Добавим на элемент
          разбиения $\Delta i$ новую точку $x'\in [x_{i-1};x_{i}]$.

          Пусть $m_{i}' = \underset{\xi\in[x_{i-1},x_{i}]}{\inf}f(\xi)$ и $m_{i}'' = \underset
            {\xi\in[x_{i}',x_{i}]}{\inf}f(\xi), \ m_{i} = \underset{\xi\in[x_{i-1};x_{i}]}{\inf}f(\xi)$,
          имеем $m_{i}\leqslant m_{i}', \ m_{i} \leqslant m_{i}''$.

          Тогда $\underline{S}(P') - \underline{S}(P) = \sum_{k=1}^{i-1}\Delta x_{k}m_{k} +
            m_{i}' | x' - x_{i-1} | + m_{i}''| x_{i} - x' | + \sum_{k=i+1}^{n}m_{k}\Delta x_{k} - \sum_{k=1}^{n}
            \Delta x_{k}m_{k} = m_{i}'| x' - x_{i-1} | + m_{i}''| x_{i} - x' | - m_{i}\Delta x_{i} \geqslant 0
            \implies \underline{S}(P') \geqslant S(P)$ (вероятно, куча индексов - неправильные).

          Аналогично доказывается для $\overline{S}(P')\leqslant\overline{S}(P)$.
    \item Пусть $P_{1},P_{2}$ - произвольные разбиения отрезка $[a;b]$.

          Возьмем разбиение $P = P_{1}\cap P_{2}$. Тогда, с одной стороны, $P$ получено из $P_{1}$ добавлением
          точек, а с другой стороны - из $P_{2}$ добавлением точек.

          Тогда $\underline{P_{i}} \leqslant \underline{S}(P)$ и $\overline{S}(P_{i})\geqslant S(P)$.

          Тогда верно, что $\underline{S}(P_{2})\leqslant \underline{S}(P)$ и $\overline{S}(P_{2})
            \geqslant \overline{S}(P) \implies \underline{S}(P_{1}) \leqslant \underline{S}(P) \leqslant
            \overline{S}(P)\leqslant \overline{S}(P_{2})$.
  \end{enumerate}
\end{proof}

\begin{effect}
  Множество нижних сумм Дарбу ограничено сверху. Множество верхних сумм Дарбу ограничено снизу.
\end{effect}

\begin{definition}[верхний/нижний интеграл Дарбу]
  Числа $\underline{\mathfrak{I}}=\sup\underline{S}(P)$ и $\overline{\mathfrak{I}}=\inf\overline{S}(P)$ называются
  \textbf{нижним} и \textbf{верхним интегралом Дарбу}.
\end{definition}

Рассмотрим множество разбиений с отмеченными точками отрезка $[a;b] \ \Re = \{(P,\xi)\}$. Построим функцию $\underline{S}:
  \Re_{\xi}\rightarrow\mathbb{R}$ и $\underline{S}((P,\xi)) = \underline{S}(P)$. Аналогично определим $\overline{S}:
  \Re_{\xi}\rightarrow\mathbb{R}$ и $\overline{S}((P,\xi)) = \overline{S}(P)$.

Таким образом сумму Дарбу можно представить как функции на множестве разбиений с отмеченными точками отрезка $[a;b]$.

\begin{theorem}[критерий интегрируемости]
  Функция $f:[a;b]\rightarrow\mathbb{R}$ интегрируема на $[a;b] \iff \underset{\lambda (P)\rightarrow0}{\lim}
    (\overline{S}(P) - \underline{S}(P)) = 0$.
\end{theorem}

\begin{proof}
  (теоремы 2.4.2)

  $"\rightarrow"$ Пусть $f\in\mathbb{R} \ ([a;b])$ (то есть интегрируема на $[a;b]$), то есть $\forall \epsilon > 0 \
    \forall (P,\xi): \ \lambda (P) < \delta \implies | \sigma_{f} ((P,\xi)) - I | < \epsilon$.

  \begin{lemma}
    $\forall P \ \underline{S}(P) = \underset{\xi}{\inf} \sigma_{f}((P,\xi))$ и $\overline{S}(P)=\underset{\xi}{\sup}
      \sigma_{f}((P,\xi))$
  \end{lemma}

  \begin{proof}
    (леммы 2.4.1)

    $\forall P \ \underline{S}(P) \leqslant \sigma_{f}((P,\xi))$.

    Покажем, что $\forall \epsilon > 0 \ \exists \xi = \{\xi_{1},\xi_{2},\ldots,\xi_{n}\}: \ \underline{S}(P) + \epsilon
      > \sigma_{f}(P,\xi)$.

    Выберем $\xi_{1},\xi_{2},\ldots,\xi_{n}: \ f(\xi_{i}) < m_{i} + \frac{\epsilon}{b-a}$.

    Тогда $\sigma_{f}(P,\xi) = \sum_{k=1}^{n}f(\xi_{k})\Delta x_{k} < \sum_{k=1}^{n}(m_{k} + \frac{\epsilon}{b-a})
      \Delta x_{k} = \sum_{k=1}^{n}m_{k}\Delta x_{k} + \frac{\epsilon}{b-a}\sum_{k=1}^{n}\Delta x_{k} = \underline{S}(P)
      + \epsilon \implies \underline{S}(P) = \underset{\xi}{\inf}\sigma_{f}(P,\xi)$.

    Аналогично для $\overline{S}(P) = \underset{\xi}{\sup}\sigma_{f}(P,\xi)$.
  \end{proof}

  $I - \epsilon < \sigma_{f}(P,\xi) < I + \epsilon, \ I - \frac{\epsilon}{2} < \sigma_{f}(P,\xi) < I + \frac
    {\epsilon}{2}$. Из леммы 2.4.1: $\underline{S}(P) + \epsilon > \sigma_{f}(P,\xi) \implies \underline{S}(P) >
    \sigma_{f} (P,\xi) - \epsilon > \sigma_{f}(P,\xi) - \frac{\epsilon}{2} \ (I = \underset{\lambda
      (P)\rightarrow0}{\lim}\sigma_{f}(P,\xi))$

  Рассмотрим $I-\frac{2\epsilon}{3} < I - \frac{\epsilon}{2} \leqslant \underline{S}(P) \leqslant \sigma_{f}(P,\xi)
    \leqslant \overline{S}(P) < \sigma_{f}(P,\xi) + \epsilon < I + \frac{\epsilon}{2} + \epsilon = I + \frac{3\epsilon}{2}
    \ (\overline{S}(P) - \epsilon < \sigma_{f}(P,\xi))$

  Тогда $I - \frac{3\epsilon}{2} < \underline{S}(P) \leqslant \overline{S}(P) < I + \frac{3\epsilon}{2}$, так как
  $\underline{S}(P) \leqslant \overline{S}(P) \implies 0 \leqslant \overline{S}(P) - \underline{S}(P),$

  \begin{center}
    $\overline{S}(P) < I + \frac{3\epsilon}{2}$ \\
    +\\
    $-\underline{S}(P) < -I + \frac{3\epsilon}{2}$
  \end{center}


  $0 \leqslant \overline{S}(P) - \underline{S}(P) < 3\epsilon \implies \underset{\lambda(P)\rightarrow0}{\lim}
    (\overline{S}(P) - \underline{S}(P)) = 0$

  $"\leftarrow"$ Пусть $\underset{\lambda(P)\rightarrow0}{\lim}(\overline{S}(P) - \underline{S}(P)) = 0$.

  Пусть $\epsilon > 0$ задана. Выберем $\delta > 0: \quad 0\leqslant\overline{S}(P) - \underline{S}(P) < \epsilon \
    \forall (P,\xi) : \ d(P) < \delta$.

  Покажем, что $\exists I = \int_{a}^{b} f(x)dx = \underset{\lambda(P)\rightarrow0}{\lim}\sigma_{f}(P,\xi)$.
  Имеем $\overline{S}(P) - \underline{S}(P) < \epsilon$ и $\underline{S}(P) \leqslant I \leqslant \overline{S}(P)$.

  Из неравенств следует, что $\overline{S}(P) < \underline{S}(P) + \epsilon \leqslant I + \epsilon, \
    \underline{S}(P) > \overline{S}(P) - \epsilon \geqslant I - \epsilon$.

  Пусть $(P,\xi)$ - произвольное разбиение: $\lambda(P) < \delta$. Тогда $I - \epsilon < \underline{S}(P)
    \leqslant \sigma_{f}(P,\xi) \leqslant \overline{S}(P) < I + \epsilon \implies I - \epsilon <
    \sigma_{f}(P,\xi) < I + \epsilon \implies | \sigma_{f}(P,\xi) - I | < \epsilon \implies I = \underset
    {\lambda(P)\rightarrow0}{\lim}\sigma_{f}(P,\xi) \implies f\in\mathbb{R} [a;b]$.
\end{proof}

\begin{definition}
  Обозначим $M_{i} - m_{i} = \underset{\xi \in \Delta i}{\sup}f(\xi) - \underset{\xi\in\Delta i}{\inf}f(\xi)
    = \underset{x_{1},x_{2}\in\Delta i}{\sup}| f(x_{1}) - f(x_{2}) | = \omega_{i} = \omega_{i}(f,\Delta i)$.

  $\omega_{i}$ называется \textbf{колебанием} функции $f(x)$ на отрезке $\Delta i$.

  $\overline{S}(P) - \underline{S}(P) = \sum_{i = 1}^{n}\omega_{i}\Delta x_{i}$
\end{definition}

\begin{effect}
  (из критерия интегрируемости)

  $f\in\mathbb{R}[a;b] \iff \underset{\lambda(P)\rightarrow0}{\lim} \sum_{i=1}^{n}\omega_{i}\Delta x_{i} = 0$
\end{effect}

\begin{theorem}[Дарбу]
  Для любой ограниченной функции $f:[a;b]\rightarrow\mathbb{R}$ выполняются равенства:

  \begin{center}
    {\large $\underline{\mathfrak{I}} = \underset{\lambda(P)\rightarrow0}{\lim}\underline{S}(P); \ \overline{\mathfrak{I}}
        = \underset{\lambda(P)\rightarrow0}{\lim}\overline{S}(P)$}
  \end{center}
\end{theorem}

\begin{lemma}
  Пусть $f:[a;b]\rightarrow\mathbb{R}$ ограничена на $[a;b]$, то есть $\exists L > 0: \ \forall x \in [a;b]
    \ | f(x) | < L$. Разбиение $P'$ получено из разбиения $P$ добавлением $m$ точек. Тогда $\overline{S}{P}
    - \overline{S}(P') \leqslant 2L\lambda(P)m$
\end{lemma}

\begin{proof}
  (леммы 2.4.2)

  Пусть $P$ - производное разбиение, $\lambda(P)$.

  Рассмотрим случай, что $P'$ получено добавлением $k$ точек на $i$-тый отрезок разбиения $P$. (график,
  посмотреть у Максима). $\overline{S}(P) - \overline{S}(P') = \sum_{j=1}^{n}M_{j}\Delta x_{j} -
    (\sum_{j=1}^{i-1}M_{j}\Delta x_{j} + \sum_{j = 1}^{k}M_{ij}'\Delta x_{ij} + \sum_{j=i+1}^{n}
    M_{j}\Delta x_{j}) = M_{i}\Delta x_{i} - \sum_{j = 1}^{k}M_{ij}'\Delta x_{ij} = M_{i}\sum_{j=1}^{k}
    \Delta x_{ij} - \sum_{j=1}^{k}M_{ij}'\Delta x_{ij} = \sum_{j=1}^{k}M_{i}\Delta x_{ij} - \sum_{j=1}^{k}
    M_{ij}'\Delta x_{ij} = \sum_{j=1}^{k}(M_{i} - M_{ij}')\Delta x_{ij} \leqslant \sum_{j=1}^{k}2L\Delta x_{ij}
    =$ (вспомним, что $\lambda(P)=\max\{\Delta x_{1},\Delta x_{2},\ldots,\Delta x_{n}\}$) $=2L\sum_{j=1}^{k}
    \Delta x_{ij} = 2L\Delta x_{i} \leqslant 2L\lambda(P)$

  Теперь, если $P'$ получено из $P$ добавлением $m$ точек, то они попадут самое большее на $m$ промежутков.
  Тогда $\overline{S}(P) - \overline{S}(P') \leqslant 2L\lambda (P)m$
\end{proof}

\begin{proof}
  (теоремы 2.4.3, Дарбу)

  $\underline{\mathfrak{I}} \overset{def}{=} \underset{P}{\sup}\underline{S}(P), \ \overline{\mathfrak{I}}
    \overset{def}{=} \underset{p}{\inf}\overline{S}(P)$

  Пусть $\epsilon > 0$ задано. Выберем разбиение $P'$ такое, что $\overline{\mathfrak{I}} + \epsilon >
    \overline{S}(P')$ (**) (определение $\inf$). Положим, что $\delta = \frac{\epsilon}{2Lm}$.

  Пусть $P$ - произвольное разбиение: $\lambda (P) < \delta$.

  Покажем, что $0 \leqslant \overline{S}(P) - \overline{\mathfrak{I}} < \epsilon$.

  Построим разбиение $P'' = P' \cup P$. Тогда $P''$ получено из $P$ добавлением $m$ точек $\implies
    \overline{S}(P) - \overline{S}(P'') \leqslant 2L\lambda(P)m$, где $L>0: \ \forall x \in [a;b] | f(x) |
    < L$. Далее, $\overline{S}(P) - \overline{S}(P'')\leqslant2L\lambda(P)m < 2Lm\delta = \frac{2Lm\epsilon}
    {2Lm} = \frac{\epsilon}{2}$. Кроме того, $P''$ получено из $P'$ добавлением некоторого количества точек.
  \begin{center}
    {\large $\overline{S}(P'')\leqslant\overline{S}(P')\overset{(**)}{<} \overline{\mathfrak{I}} + \frac
        {\epsilon}{2}\implies \overline{S}(P'')-\frac{\epsilon}{2}<\overline{\mathfrak{I}}$}
  \end{center}
  Рассмотрим $0\leqslant\overline{S}(P) - \overline{\mathfrak{I}} < \overline{S}(P) - \overline{S}(P'') +
    \frac{\epsilon}{2} < \frac{\epsilon}{2} + \frac{\epsilon}{2} = \epsilon$
\end{proof}

\subsection{Классы интегрируемых функций}

\begin{theorem}[интегрируемость непрерывных функций]
  Пусть $f:[a;b]\rightarrow\mathbb{R}$ непрерывна на $[a;b]\implies f$ - интегрируема на $[a;b]$
  , то есть $f\in\mathbb{R}[a;b]$.
\end{theorem}

\begin{proof}
  (теоремы 2.4.4)

  Так как $f$ - непрерывна на $[a;b]\implies f$ - равномерно непрерывна на $[a;b]$. Это значит,
  что если $\epsilon>0$ задано, то $\exists\delta>0: \ \forall x_{1},x_{2}\in [a;b]: \
    | x_{1} - x_{2} | < \delta \implies | f(x_{1}) - f(x_{2}) | < \frac{\epsilon}{b - a}$.

  По критерию интегрируемости: $f\in\mathbb{R}[a;b] \iff \underset{\lambda(P)\rightarrow0}{\lim}
    (\overline{S}(P) - \underline{S}(P)) = 0 \ \forall (P;\xi)$ - разбиение.

  $\overline{S}(P) - \underline{S}(P) = \sum\omega_{i}\Delta x_{i}$, где $\omega_{i} = \underset
    {x_{1},x_{2}\in\Delta i}{\sup} | f(x_{1}) - f(x_{2}) |$.

  $\overline{S}(P) = \sum M_{i}\Delta x_{i}, \ M_{i} = \underset{\xi\in\Delta x_{i}}{\sup}f(\xi)$.

  $\omega_{i} = M_{i} - m_{i} = \underset{\xi\in\Delta i}{\sup}f(\xi) - \underset{\xi\in\Delta i}
    {\inf}f(\xi) = \underset{x_{1},x_{2}\in\Delta i}{\sup}| f(x_{1}) - f(x_{2}) |$.

  $\overline{S}(P) - \underline{S}(P) = \sum M_{i}\Delta x_{i} - \sum m_{i}\Delta x_{i} =
    \sum \omega_{i}\Delta x_{i}$.

  Таким образом критерий интегрируемости: $f$ - интегрируема на $[a;b] \iff \underset{\lambda
      (P)\rightarrow0}{\lim}\sum\omega_{i}\Delta x_{i} = 0$, то есть $\forall \epsilon > 0 \
    \exists \delta > 0: \ \forall (P;\xi): \ \lambda(P)<\delta\implies 0 \leqslant\sum\omega_{i}
    \Delta x_{i} < \epsilon$.

  Пусть $\epsilon > 0$ задано. Возьмем $(P;\xi)$ - разбиение такое, что $\lambda(P) < \delta$.
  Тогда $\sum\omega_{i}\Delta x_{i} = \sum \underset{x_{1},x_{2}\in\Delta i}{\sup}
    | f(x_{1}) - f(x_{2}) |\Delta x_{i}\leqslant \sum \frac{\epsilon}{b - a} \Delta x_{i} =
    \frac{\epsilon}{b - a}\sum \Delta x_{i} = \frac{\epsilon}{b - a}(b-a) = \epsilon$
\end{proof}

\begin{theorem}[интегрируемость функций с конечным числом точек разрыва]
  Пусть $f:[a;b]\rightarrow\mathbb{R}$ - ограничена и имеет на $[a;b]$ конечное число точек
  разрыва. Тогда $f\in\mathbb{R}[a;b]$ интегрируема на $[a;b]$.
\end{theorem}

\begin{proof}
  (теоремы 2.4.5)

  Пусть $L > 0: \ \forall x \in [a;b] \ | f(x) | < L$ (ограничена). Пусть $f$ имеет $k$ точек
  разрыва на $[a;b]$.

  Пусть $\epsilon > 0$ задано. Возьмем $\delta_{1} = \frac{\epsilon}{16Lk}$. Для каждой точки
  разрыва построим $\delta_{1}$-окрестность.

  Пусть $U$ - множество таких окрестностей. $U$ - открытое множество. Рассмотрим
  $V = [a;b]\setminus U \implies V$ - замкнутое (так как его дополнение открытое). Из того,
  что $V$ - ограничено и замкнуто $\implies V$ - компактное. Функция $f$ - непрерывна на $V
    \implies$ из того, что $V$ - компактно и $f$ - непрерывна на $V \implies f$ - равномерно
  непрерывна на $V\implies \forall\epsilon>0 \ \exists\delta_{2}>0: \ \forall x_{1},x_{2}\in V:
    \ | f(x_{1}) - f(x_{2}) | < \frac{\epsilon}{2(b-a)}$.

  Положим, что $\delta = \min\{\delta_{1},\delta_{2}\}$. Пусть $P$ - произвольное разбиение
  отрезка $[a;b]: \ \lambda(P) < \delta$.

  Рассмотрим $\sum \omega_{i}\Delta x_{i} =  \sum'\omega_{i}\Delta x_{i} + \sum''\omega_{i}
    \Delta x_{i} \leqslant | \sum'$ берется по всепм отрезкам разбиения, $k$-тые пересекаются
  с $U$, $\sum''$ - по всем остальным $| \leqslant \sum'\omega_{i}\Delta x_{i} + \sum''
    \frac{\epsilon}{2(b-a)}\Delta x_{i} \leqslant 2L2\delta_{1}k + \frac{\epsilon}{2(b-a)}\sum''
    \Delta x_{i} < \frac{4Lk\epsilon}{8Lk} + \frac{\epsilon}{2(b-a)}(b-a) = \frac{\epsilon}{2}
    + \frac{\epsilon}{2} = \epsilon$.

  Дополнение: $(\overline{S}(P) - \overline{S}(P') \leqslant
    2L\lambda(P)m, \ \sum M_{i}\Delta x_{i} - \sum M_{i}' \Delta x_{i})$.
  $\sum'\omega_{i}\Delta x_{i} = \sum\underset{x_{1},x_{2}\in\Delta i \cap k}{\sup}
    | f(x_{1}) - f(x_{2}) |\Delta x_{i} \leqslant 2L2\delta_{1}k$

\end{proof}

{\Large ГРАФИКИ НАДО НАРИСОВАТЬ}

\begin{theorem}[интегрируемость монотонных функций]
  Пусть $f:[a;b]\rightarrow\mathbb{R}$ - монотонна на $[a;b]\implies f$ - интегрируема на $[a;b]$.
\end{theorem}

\begin{proof}
  (теоремы 2.4.6)

  Пусть $f$ - не убывает на $[a;b]$. Пусть $\epsilon>0$ задано. Возьмем $\delta = \frac{\epsilon}
    {f(b) - f(a)}$. Тогда, если $P$ - произвольное разбиение $[a;b]: \ \lambda(P)<\delta$, то
  $\sum\omega_{i}\Delta x_{i} \overset{monoton.}{=} \sum (f(x_{i}) - f(x_{i-1}))\Delta x_{i} <
    \delta \sum (f(x_{i}) - f(x_{i-1})) = \delta (f(b) - f(a)) = \epsilon$.
\end{proof}

\subsection{Свойства интегрируемых функций}

\begin{theorem}
  Пусть $f\in\mathbb{R}[a;b], \ g\in\mathbb{R}[a;b]$. Тогда:
  \begin{enumerate}
    \item $f\pm g \in R[a;b]$.
    \item $\alpha f \in R[a;b], \ \alpha \in \mathbb{R}$.
    \item $f*g\in R[a;b]$.
    \item $|f|\in R[a;b]$, при этом:
          \begin{itemize}
            \item $\int_{a}^{b}(f\pm g)dx = \int_{a}^{b}f(x)dx \pm \int_{a}^{b}g(x)dx$
            \item $\int_{a}^{b}\alpha f(x)dx = \alpha \int_{a}^{b}f(x)dx$
            \item $|\int_{a}^{b}f(x)dx | \leqslant \int_{a}^{b}|f(x)|dx$
          \end{itemize}
  \end{enumerate}
\end{theorem}

\begin{proof}
  (теоремы 2.4.7)

  \begin{enumerate}
    \item $\int_{a}^{b}(f(x)\pm g(x))dx = \underset{\lambda(P)\rightarrow0}{\lim}\sum(f(\xi_{i})
            \pm g(\xi_{i}))\Delta x_{i} = \underset{\lambda(P)\rightarrow0}{\lim}\sum f(\xi_{i})\Delta
            x_{i} = \int_{a}^{b}f(x)dx + \int_{a}^{b}g(x)dx$.
    \item Аналогично.
    \item Покажем, что если $f\in R[a;b]$, то $f^{2}\in R[a;b]$. Рассмотрим $| f^{2}(x_{1}) -
            f^{2}(x_{2}) | = | (f(x_{1}) - f(x_{2}))(f(x_{1}) - f(x_{2})) | \leqslant | f(x_{1}) -
            f(x_{2}) |(|f(x_{2})| + | f(x_{2}) |) < 2L| f(x_{1}) - f(x_{2}) |$, где $L>0: \ \forall x
            \in [a;b] \ | f(x) | < L$ (интегрируема $\implies$ ограничена).

          Пусть $P$ - произвольное разбиение. Пусть $\epsilon > 0$ задано. Возьмем $\delta>0$ и
          $P: \ \lambda(P) < \delta \quad \omega_{i}(f^{2},\Delta_{i}) \leqslant 2L\omega_{i}
            (f,\Delta_{i})$
  \end{enumerate}
\end{proof}


\end{document}